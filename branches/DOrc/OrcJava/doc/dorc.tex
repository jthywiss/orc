\documentclass[10pt,letterpaper]{article}

\usepackage{times}
\usepackage{epsfig}
\usepackage{graphicx}
\usepackage{amsmath}
\usepackage{amssymb}

\begin{document}

%%%%%%%%% TITLE
\title{Distributed \textsc{Orc} (\textsc{DOrc})}

\author{Adrian Quark\\
{\tt\small quark@mail.utexas.edu}}

\maketitle
% \thispagestyle{empty}

\section{Introduction}

\textsc{Orc} is a language designed for orchestrating distributed computations,
but it has a significant limitation: all distributed communication must be
mediated by site calls, and currently sites cannot be implemented within
\textsc{Orc}.  This causes problems in two situations:
\begin{itemize}
\item In order to implement a trivial distribution task (for example, ``open a
yes/no dialog box on another computer and get the user's response''), the
programmer must implement a site in Java and design a communication protocol to
connect to it, which may be a lot of work for such a simple task.
\item A large program written in \textsc{Orc} cannot be easily broken into
parts and run in a distributed manner. Again, it would be necessary for the
programmer to implement sites to run on each distributed system and explicitly
handle all the communication between \textsc{Orc} processes.
\end{itemize}

The goal of this project is to solve these problems by introducing a new syntax
for distributed expressions. The programmer simply annotates an existing
\textsc{Orc} program to indicate where (on which computer in the distributed
system) each sub-expressions should be run, and the \textsc{Orc} interpreter
transparently handles all distributed communication. ``Remote'' sub-expressions
use remote computing resources, as does a site call, but otherwise have all the
properties of regular \textsc{Orc} expressions, including:
\begin{itemize}
\item The same \textsc{Orc} syntax
\item The ability to call functions and refer to variables defined in the
surrounding \textsc{Orc} program
\item The ability to publish multiple values
\item The ability to participate in asymmetric composition
\end{itemize}

\section{Usage}

\textsc{DOrc} introduces one new type of value and one new syntactic construct
to the language. The new value type is a ``server'', which corresponds to a
logical computer in a distributed computation\footnote{In distributed computing
literature this is usually called a ``node'', but I have chosen ``server'' to
avoid confusion with the term ``node'' used in the context of the \textsc{Orc}
graph-based implementation.}. The new syntactic construct is the ``remote
expression'', which specifies that an expression should be evaluated on a
specific server.

\subsection{Servers and Meta-Servers}

A ``server'' is a logical concept; it is entirely possible to have multiple
\textsc{DOrc} servers running on the same physical computer.

Every server participates in one and only one distributed computation, and all
servers in the computation share the same environment (variables and function
definitions). This restriction is inforced by making it impossible to obtain a
reference to a server outside of the computation in which it participates. In
the context of evaluating a specific expression across two servers, one server
(which initiated the computation) is the master and the other server (which is
performing the evaluation) is the slave.  However in the context of the program
as a whole, each server may participate in the evaluation of several
expressions simultaneously, and therefore there may be no clear master-slave
structure to the overall computation.

To create a new server, a \textsc{DOrc} program calls a meta-server, which is
simply an ordinary \textsc{Orc} site which returns servers. Each meta-server
has a known global name, so that any \textsc{DOrc} process may request a new
server from it. Typically a meta-server corresponds to a specific physical
computer and produces servers which evaluate expressions on that computer, but
it is entirely possible to implement a meta-server which acts on behalf of a
pool of physical computers, returning servers which may evaluate expressions on
any member of the pool.

Currently meta-servers are implemented as standard Java RMI servers, which can
be contacted using the built-in site \texttt{Remote}:
\begin{verbatim}
val metaserver = Remote('rmi://address:port/path')
\end{verbatim}
Incidentally, the \texttt{Remote} site is not specific to \textsc{DOrc}, it is
just a regular \textsc{Orc} site and can be used to connect to any Java RMI
server. The ``address:port'' portion of the URL specifies where the
\texttt{rmiregistry} server can be found, and the ``path'' specifies which
object in the registry to contact.

To get a new server from a metaserver, use the method \texttt{newServer}:
\begin{verbatim}
val server = metaserver.newServer()
\end{verbatim}
The \texttt{server} can be used in distributed expressions, described in the
next section.

Use the site \texttt{Local} to get a reference to the current \textsc{DOrc}
server. This is useful if you want to execute an expression on a
remote server which executes some inner expression back on the local server.

\subsection{Distributed Expressions}

A remote expression is of the form \texttt{f @ r} where \texttt{f} is an
arbitrary expression and \texttt{r} is a variable which holds a reference to a
\textsc{DOrc} server. The \texttt{@} operator has higher precedence than any
other operator. As with site calls, \texttt{r} may be an expression instead of
a variable, in which case it is equivalent to \texttt{f @ x <x< r}.

The meaning of such an expression is: when the value of \texttt{r} is
available, evaluate the expression \texttt{f} on the server specified by the
value of \texttt{r}. The precise semantics of this expression are discussed in
a later section.

\subsection{Shared Sites}

For the most part, you can think of site calls made in remote expressions as
equivalent to the same site calls made in local expressions. For example, if
you write a value to a channel in a remote expression, you can read that value
from the channel locally.

Of course, if all sites were like this, then distributed programs would be
pretty boring because they could never affect the remote server. So many sites
are understood to operate relative to whatever server they are called on (the
``current server''). As a rule of thumb, any built-in site with a global name
affects the current server, while any dynamically-created site affects the
server where it was created. Some specific examples:
\begin{itemize}
\item Any site which creates a new site allocates it on the current
server: \texttt{Buffer}, \texttt{Cell}, \texttt{Ref}, \texttt{SyncChannel}, etc.
\item Printing sends output to the current server's console: \texttt{print},
\texttt{println}. If you want to send output to a particular console, use the
\texttt{Printer} site to create a printer object which will print to the
console where it was created.
\item \texttt{Localhost} and \texttt{Local} refer to the current server, because that's useful.
\end{itemize}

\subsection{Examples}

Let us consider some examples of distributed programs. These are not very
interesting because they are all equivalent to similar non-distributed
programs, but they serve to illustrate the variety of communication which may
occur between distributed servers.  In the following examples, I will assume
the existence of sites \texttt{c.put} and \texttt{c.get}, which put and get to
an asynchronous buffer. If the buffer is empty, \texttt{c.get} waits to return
until a new item is placed in the buffer by \texttt{c.put}. I will also assume
that a remote server has been created and is available in the variable
\texttt{r}.

\begin{description}
\item[\texttt{1 @ r}] is the simplest remote expression. It evaluates the constant
\texttt{1} at the remote server \texttt{r} and finally publishes the value
\texttt{1} back to the local server.

\item[\texttt{(1 + 2)@r}] is a slightly more complex expression. It actually carries
out some computation on \texttt{r}, evaluating \texttt{1+2} and publishing the
result.

\item[\texttt{(Rtimer(1) | Rtimer(2))@r}] will start two \texttt{Rtimers} on
\texttt{r} and publish values after 1 and 2 time units. This example
illustrates that, unlike a site call, a remote expression may publish multiple
values.

\item[\texttt{Rtimer(1)@r | Rtimer(2)@r}] gives the exact same result as
\texttt{(Rtimer(1) | Rtimer(2))@r}, modulo timing concerns discussed in the
next section. This example illustrates that it is possible to use a server
multiple times. In this case, the server is used to execute two computations
concurrently, but it could also be used to execute multiple computations in
sequence, or with any degree of overlap in time.

\item[\texttt{c.get() | c.put(1)@r}] places a value on the buffer at the remote
server, and retrieves it locally. Distributed expressions can communicate via
sites just like local expressions.

\item[\texttt{(c.put(1) >> let(x))@r <x< Rtimer(1)}] evaluates the \texttt{Rtimer}
locally in parallel with the \texttt{c.put(1)} remotely. When the remote
server reaches the \texttt{let(x)}, it must wait for the local server to
publish a value to \texttt{x} before it can proceed.

\item[\texttt{let(x) <x< (Rtimer(1) | Rtimer(2)@r)}] evaluates one \texttt{Rtimer}
locally and one remotely. The local \texttt{Rtimer} publishes a value first,
and when it does, all further computation of the parallel remote expression is
terminated and no value will be published from it.
\end{description}

\section{Semantics}

The semantics of a remote expression are closely related to the timing
semantics of \textsc{Orc}. Traditionally, \textsc{Orc} sites are classified
into immediate sites, whose values are published at precisely-defined times or
not at all, and non-immediate sites, whose values may be published after
arbitrary delay. \texttt{let} and \texttt{Rtimer} are examples of immediate
sites. The fact that these sites are immediate means that the expression
\texttt{Rtimer(1) >> 1 | Rtimer(2) >> 2} is guaranteed to publish the values 1
and 2 in that order.  This requirement is problematic for a distributed
implementation, because starting a distributed expression may involve arbitrary
delay.

The simplest solution would be to discard the concept of immediate sites. If
all sites are allowed to wait an arbitrary amount of time before publishing a
value, any delay introduced by remote communication may be attributed to the
delay in some site returning a value.  Unfortunately, this means that it is no
longer possible for a program to rely on the order that values are published by
any expression.  Whether this is a problem for typical \textsc{Orc} programs
remains a subject for future study.

A slightly more refined solution may be possible.  The delay introduced by
distributed communication affects the semantics only if it is observable.
Therefore, it suffices to ensure that the chain of causal relationships
connecting any local event to any locally observable result of evaluation on a
remote server (the remote server publishing a value or calling a stateful site)
includes some non-immediate site to which the delay in distributed
communication can be attributed.

I believe, but have not proven, that the expression \texttt{f@r} is exactly
equivalent to \texttt{LET() >> f >x> LET(x)}, where \texttt{LET} is a
non-immediate form of \texttt{let}, and all stateful sites (such as buffers)
are also considered non-immediate.  The reasoning behind this is as follows:
the local node can only communicate with the remote node via the initiation of
the expression, through a stateful site, or through a future. The delay in the
initiation of the expression is accounted for by the non-immediate site call
\texttt{LET()}. Communication through stateful sites is also subject to the
non-immediacy of these sites. The publication of a where value cannot by itself
convey any timing information, and so can only be given a precise time relative
to some event observed via another means.

A good example to illustrate the problems with distributed semantics is:
\begin{verbatim}
( let(a) <a< (let(x) | let(y)) )@r
    <x< (Rtimer(1) >> 1)
    <y< (Rtimer(2) >> 2)
\end{verbatim}
This example shows that even with the \texttt{LET}-based semantics described in
the previous paragraph, the distributed communication mechanism must guarantee
in-order delivery to ensure that the correct value is published by
\texttt{let(a)}.  This problem is subtle enough that a proof of the correctness
of the distributed implementation must be provided before the programmer relies
on any semantics involving immediate sites.

\section{Implementation}

I had four goals for \textsc{DOrc}:
\begin{itemize}
\item Don't hurt non-distributed computation
\item Optimize to avoid communication
\item Be conservative with optimization
\item Keep as much of existing architecture and code as possible
\end{itemize}

I started from the assumption that an active token must reside on the server
which is processing it, then identified related objects which can safely be
copied between servers. Such objects include the environment and node graph,
which are immutable, but not mutable sites and group cells (futures). Any
object which cannot be copied between servers must be shared via a remote
reference.

This implies that the following events may require distributed communication:
\begin{itemize}
\item Activating a token at the start of a remote expression
\item Publishing a value from a remote expression
\item Sending arguments to a mutable site
\item Returning a value from a mutable site
\item Assigning a value to a group cell
\item Notifying a token waiting on a group cell that a value is available
\item Killing tokens associated with a group cell which has received a value
\end{itemize}

\begin{figure}[t]
\begin{center}
\includegraphics[width=0.9\linewidth]{orc_communication.pdf}
\end{center}
\caption{\textsc{Orc} communication}
\label{fig:orc_communication}
\end{figure}

Figure~\ref{fig:orc_communication} shows communication in the non-distributed
implementation, where each box represents a class and each arrow represents a
reference via which messages can be sent. I have labeled those arrows whose
purpose may not be obvious to someone with only a casual familiarity with the
implementation.

\begin{figure}[t]
\begin{center}
\includegraphics[width=0.9\linewidth]{dorc_communication.pdf}
\end{center}
\caption{\textsc{DOrc} communication}
\label{fig:dorc_communication}
\end{figure}

In contrast, Figure~\ref{fig:dorc_communication} shows communication in the
distributed implementation. Solid arrows are local references, while dashed
arrows are references to objects which may live on remote servers. You will
notice two significant changes, which are discussed in more detail in later
sections:
\begin{itemize}
\item \texttt{GroupCell} has been split into \texttt{Group} and \texttt{Cell}
\item \texttt{Site} has been split out of \texttt{Value}
\end{itemize}

\subsection{Java RMI}

For the actual implementation of distributed communication, I chose Java RMI,
which has several benefits:
\begin{itemize}
\item Objects on remote servers can be called using the same syntax as local
objects.
\item Arguments to remote methods are automatically serialized. Objects which
are registered with RMI as remote objects are serialized to a remote reference
which allows remote method calls, while all other objects are copied and not
shared between servers.
\item The RMI system provides garbage collection for remote references via
reference counting. Unfortunately, RMI garbage collection does not collect
cycles. It remains to be seen whether this is a problem for typical
\textsc{DOrc} programs.
\end{itemize}
Automatic serialization was by far the biggest benefit: the \textsc{Orc}
environment may contain arbitrarily-complicated values, and implementing my own
serialization routine for such values would be time consuming and also
complicate future work on the \textsc{Orc} implementation.

\subsection{Remote Expressions}

\textsc{DOrc} remote expressions are handled exactly like \textsc{Orc} function
calls, with only two significant differences. First, since remote expressions
are always executed in the environment of the caller, there is no need to
switch the environment of the token when it is moved to the body of the remote
expression. Second, instead of simply moving the token to the body of the
expression, it is serialized together with the expression DAG and sent to the
remote server to be evaluated.

\subsection{Remote Site Calls}

The \textsc{Orc} environment itself is immutable and can safely be copied
between servers. The same is true for the majority of primitive values used in
\textsc{Orc} programs, including numbers, lists, tuples, and strings. However
there are a few mutable sites, such as buffers, which must be shared between
servers.

This sharing is implemented by translating references in the environment to
such mutable sites into remote references when the environment is copied to a
remote server. If the remote server tries to call such a site, it will be a
remote site call. When making a remote site call, the arguments to the site and
a remote reference to the return token must be serialized and passed to the
site via distributed communication.  When the site is ready to return a value,
it uses the remote reference to send the value back to the token on the local
site.

\textsc{DOrc} separates sites into an immutable reference value
(instance of {\tt orc.run\-time.values.Site}) and a site implementation (instance
of subclass of {\tt orc.run\-time.sites.Site}) which may encapsulate mutable
state.  Furthermore the interface \texttt{Passed\-By\-Value\-Site} is used to
mark site implementations which should be copied, either because they are pure
or because they should be interpreted relative to the current server.

\subsection{Remote Futures}

The asymmetric combinator requires careful implementation in the distributed
case. The naive approach is to simply allow a group cell to always be handled
as a remote reference. This is bad for two reasons.

First, a token must check its group cell every time it is processed in order to
ensure the group is still alive. If the group cell is a remote reference this
introduces a significant overhead to every token processing step. This can be
easily rectified by introducing a local proxy for the remote group cell. The
token only has to check the local proxy, which will be automatically notified
by the remote group cell when the group is killed.

Second, a token may communicate with a group cell to check if it has received a
value, to wait on that value, and finally to be notified when the value
arrives. If multiple tokens on the same server all need the value, they will
make redundant requests: they will all request the value separately from the
group cell, even though some other token on the same server may already know
the value. My solution is to introduce a local cache on each server which acts
as an intermediary between tokens and group cells. Instead of every token
asking the group cell directly for a value, they ask the local cache. If the
local cache does not have the value, it asks the group cell on their behalf.
If necessary, the local cache waits for the value and is notified by the remote
group cell when a value is ready, so that it in turn may notify all of the
tokens waiting on its server.

One important question is whether these optimizations are correct.
Specifically, when a group cell is killed, there may be a race condition due to
the delay in notifying the tokens working on that group cell. How do we know
this delay will not cause problems?  Correctness requires that we must
guarantee two things:
\begin{enumerate}
\item The group publishes exactly one value.
\item It should not be possible to observe progress of tokens in the group
after the value is published.
\end{enumerate}

The second requirement deserves further explanation. What does it mean to
``observe progress \ldots after the value is published''? In a semantics with
no immediate mutable sites (such as that proposed for \textsc{DOrc}), this
simply means that there cannot exist a causal relationship between the
publishing of a value and some event which occurs on the right-hand side after
the value is published. In other words, a token on the right-hand side should
not observe any event from the left-hand-side that depends on the value being
published, and a token on the left-hand side should not observe any event from
the right-hand side that occurs after the value is published.

The following example illustrates both points:
\begin{verbatim}
let(x) >> d.put(1)
    <x< c.get()
      | (c.put(1) >> d.get())@r
    <c< Buffer()
    <d< Buffer()
\end{verbatim}
If requirement (1) is not guaranteed, \texttt{x} may receive a value twice. If
requirement (2) is not guaranteed, then the remote node may observe that the
left-hand side of the asymmetric combinator has made progress by receiving a
value via \texttt{d}.

Note that with no immediate sites, the following program may legally produce a
value:
\begin{verbatim}
let(x) >> c.get()
    <x< c.get()
      | (c.put(1) >> c.put(2))@r
    <c< Buffer()
    <d< Buffer()
\end{verbatim}
The explanation is that even though \texttt{c.put(1)} ultimately causes
\texttt{x} to receive a value, there may be an arbitrary delay between sending
the value to the channel and that value being received by the first
\texttt{c.get()}, during which \texttt{c.put(2)} may still legally run. This
would only be a problem if the program could observe definitively that
\texttt{x} had received a value before \texttt{c.put(2)} started running, which
is covered by requirement (2) above.

In \textsc{DOrc} the two requirements are guaranteed easily:
\begin{enumerate}
\item is guaranteed by a mutex on the group cell, which ensures only one value
may be published at a time and ignores any attempts to publish values after the
first has been published
\item is guaranteed by waiting until all tokens in the right-hand side have
been notified of group death to proceed with the left-hand side
\end{enumerate}

\subsection{Deadlock Freedom}

A valuable property of \textsc{Orc} is that any program which does not use
mutable sites is free from deadlock. A proof sketch: in the absence of mutable
sites, communication between any two concurrent computations is one-way. There
can be no communication between the left and right sides of \texttt{|},
\texttt{>x>} does not enable communication between concurrent instances of the
right sub-expression, and \texttt{<x<} only allows communication from the right
side to the left. Therefore if one side of such a combinator depends on the
other for progress, the converse cannot be true, so deadlock is impossible.

\textsc{DOrc} does not change the core combinators, so this statement
remains true provided the \textsc{DOrc} implementation itself is free from
deadlock. I have not proven this latter result, but it should be fairly
straightforward to do so. Since the \textsc{DOrc} implementation uses
essentially the same communication structure as the original \textsc{Orc}
implementation, assuming the original implementation is free from deadlock,
the \textsc{DOrc} implementation only has to worry about distributed deadlock.
Distributed deadlock can be avoided by ensuring that all distributed messages
are non-blocking if they may need to acquire a lock. I accomplish this by
running key distributed remote procedure calls in a separate thread, which
makes them non-blocking with respect to the main thread.

With arbitrary stateful sites, deadlock freedom is clearly not guaranteed, in
either the distributed or non-distributed case. Further research is needed to
determine whether there is some useful subset of sites or structures for
expressions which preserve deadlock freedom while still enabling more
interesting computation.

\section{Future Work}
\label{sec:conclusion}

Significant future work remains in improving the efficiency of the current
implementation. There are three key areas for improvement:
\begin{itemize}
\item \texttt{LTimer} is broken. It should be possible to fix this using a
three-phase commit protocol so that all servers can agree when a logical time
unit has elapsed, but implementing this correctly without race conditions may
be tricky, especially given that servers may engage in arbitrary communication
patterns.
\item Remote references are not unpacked when sent back to their originating
node. In other words, if a server creates a channel and passes it to another
server, which then passes it back, the original server will end up with a
remote reference to a local object. Anytime the server uses this object, it
will incur needless overhead serializing arguments and transmitting them via
the RMI protocol. The solution to this problem involves creating a global
identifier for every remote object which is unchanged as the object is passed
between servers.  Each local server can keep a cache of identifiers for remote
objects it originated, and when it receives such a remote object it can replace
it with its local implementation.
\item Whenever a remote expression is evaluated, the entire expression is
copied to the remote server. Since the expression is immutable, this copying
may be unnecessary if the remote expression was evaluated before. The solution
is to copy the entire DAG to the remote server when it is used for the first
time, and then for subsequent uses it can refer to its local copy of the DAG
rather than being sent a new one. One complication is that this requires a
global identifier for the node at the start of a remote expression, so that the
remote server can be told where to evaluate a token.
\item Finally, the entire environment is copied to the remote server whenever a
remote expression is evaluated. If the expression only needs a small part of
the environment this is very wasteful. Since infrastructure already exists to
track free variables, it should be straightforward to identify the free
variables in an expression and only copy the portion of the environment they
refer to, transitively.
\end{itemize}

Another important area for future work is in proving the correctness of the
distributed implementation, and providing stronger guarantees about timing. My
intuition is that this requires a type system which can be used to prove
assertions about sites, so that it is possible to automatically verify whether
important semantic properties might be violated by distributing an expression.

\end{document}
