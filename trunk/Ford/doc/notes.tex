\documentclass{article}[12pt]

\begin{document}
\title{Forensic \textsc{Orc} Debugger (\textsc{Ford}) Notes}\author{Adrian Quark}\maketitle

\section{Traces}

Preliminaries:
\begin{itemize}
\item An Orc computation consists of sequences of {\em events} occurring in
logical {\em threads}.
\item Threads form a tree, where each thread is connected to the thread which
spawned it.
\end{itemize}

The following minimal events must be traced to deterministically replay a computation:
\begin{description}
\item[root] The root of the thread tree.
\item[fork] A fork event. By convention, the ``left'' branch of the fork (as
dictated by the syntax) is evaluated by the same thread, while the right branch
is evaluated by a new thread.
\item[resume] A site call returning. The order in which site calls return (and
the values they return) is the primary source of non-determinism in Orc
computations.
\end{description}

A trace is a DAG of events:
\begin{itemize}
\item Edges are only allowed to point ``back'' in time, i.e. from an event to
one which preceeded it in time.
\item Each non-\texttt{root} event has a \texttt{parent} edge which
points to its immediate predecessor in time in the same or parent logical
thread.
\item The \texttt{fork} event is the only event which may have more than one
incoming \texttt{parent} edge.
\item Each non-\texttt{root} event has a \texttt{thread} edge pointing to the
closest \texttt{parent} ancestor \texttt{fork} or \texttt{root} event. This
serves as the thread identifier for events in the same logical thread.
\item Each \texttt{resume} event has an \texttt{after} edge pointing to the
\texttt{resume} event which preceeded it in time. This serves to encode {\em
possible} causal relationships between site calls in different threads; since
we don't have any knowledge of site internals, we have to assume that all
temporal relationships may reflect causal relationships.
\end{itemize}

A trace file contains a sequence of serialized events from the same trace,
followed optionally by one or more indices.

Events are serialized into the trace file in the order in which they occur in
time.  \texttt{after} and \texttt{parent} edges are represented implicitly by
the serialization order; all other edges are explicitly represented by
references to already-serialized events.

\section{Events}

\begin{description}
\item[sleep] Blocking to wait for a future to receive a value.
\item[wake] Waking after a future has received a value.
\item[enter] Entering an expression.
\item[leave] Leave an expression (via a publication).
\end{description}

\section{User Interface}

The debugger will be text-based, for rapid development. The concepts should be straightforward
to extend to a graphical debugger.

\begin{itemize}
\item The interface will consist of a thread list followed by a command prompt.
\item The thread list will display a "primary" thread together with zero or
more "secondary" threads, each with a unique number.
\item The command prompt will be used to enter commands such as:
	\begin{itemize}
	\item select a primary thread
	\item hide a thread
	\item hide all secondary threads
	\item step forwards or backwards one event in the primary thread
	\item display contextual information about the primary thread's current event
	\item search for an event
	\end{itemize}
\item Each thread in the thread list will include the following information:
	\begin{enumerate}
	\item The source location and line of code which includes the start of the expression the thread is evaluating
	\item The expression the thread is evaluating, elided if necessary to fit on a single line
	\item The state of the thread (blocked, running, or terminated)
	\end{enumerate}
\item The primary thread will also display several lines of source code around the current expression, for context.
\item Stepping involves simulating events from the primary event stream until
the next (or previous) event for the primary thread is reached. Note that these
events may include events in other threads, although such events will never
change the primary thread directly.
\item When any thread in the thread list forks, the new thread is appended to the thread list as a secondary thread.
\item When any thread in the thread list terminates, it is removed from the
thread list after the next forward step command.
\item When stepping backwards past the point a secondary thread was created, it is removed from the thread list.
\item When stepping backwards past the point the primary thread was created, it is replaced by the creating thread.
\item Contextual information for the current event (accessible via the command line) includes:
	\begin{itemize}
	\item The value of any bound variable
	\item For any unbound variable (future), the event which originated the future
	\item For a blocked thread, the event which originated the future
	\item For a terminated thread, the event which terminated it
	\end{itemize}
\item To search for an event, the user enters a query and is presented with a
list of matching events. The user chooses one and it is selected as the primary
thread. This feature will almost certainly require indices in the trace file
for efficient operation.
\end{itemize}

\end{document}
